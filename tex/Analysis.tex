\subsection{Finding Activity from Fitted Peaks}
The activity from nuclear decay follows an exponential form as follows:
\begin{equation} \label{eq: A(t)}
  A(t) = A_0e^{-λt}
\end{equation}
with $A_0$ being the initial activity and $λ$ the decay constant. The decay constant for each isotope is already known to be $λ_{108} = 2.382$ minutes and $λ_{110} = 24.56$ seconds for $^{108}\text{Ag}$ and $^{110}\text{Ag}$ respectively. The total number of decays from a delayed time $t_d$ after end of irradiation, with a counting time of $t_c$ is given by:
\begin{equation} \label{eq: N_D}
  N_D = ∫_{t_d}^{t_d + t_c} A(t) \ \mathrm{d}t.
\end{equation}
Solving \cref{eq: N_D} gives:
\begin{equation}
  N_D = \frac{A_0}{λ}e^{-λt_d}\left(1 - e^{-λt_c}\right)
\end{equation}
combined with \cref{eq: A(t)} gives: 
\begin{equation} \label{eq: N_D2}
  N_{D} = \frac{A(t_d)}{λ}\left(1 - e^{λt_c}\right).
\end{equation}

We are not able to count every decay, as the detector has a finite efficiency $ϵ$, and different photon energies have different intensities $I_{γ}$. The number of counts $N_C$ is then given by:
\begin{equation} \label{eq: N_C}
  N_C = ϵI_{γ}N_D 
\end{equation}

Combining \cref{eq: N_C} and \cref{eq: N_D2} we can express the activity at a delayed time $t_d$ as:
\begin{equation} \label{eq: A(t_d)}
  A(t_d) = \frac{N_Cλ}{ϵI_{γ}\left(1 - e^{-λt_c}\right)}.
\end{equation}

With multiple successive measurements of the number of counts $N_C$ at different delayed times $t_d$, we are able to fit the data to the exponential form and find the initial activity $A_0$ at the end of irradiation. 
